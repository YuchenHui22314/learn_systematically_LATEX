\documentclass{ctexart}


%%%%%%%%%%%%%%%%%%%%%%%%%%%%%%%%%%%%PACS%%%%%%%%%%%%%%%%%%%%%%%%%%%%%%%%
\usepackage{tabularx, makecell, multirow }
\usepackage{syntonly}
%hpyerlien
\usepackage{hyperref}
%utf8
%\usepackage[utf8]{inputenc}
%Chinese
\usepackage[]{ctex}
%to solve footnote too high 
%(this helps fix footnote to bottom of the page)
\usepackage[bottom]{footmisc}
%array: aimed at tabular
\usepackage{array}
% sanxianbiao
\usepackage{booktabs}
% graphics
\usepackage{graphicx}
% amsmath
\usepackage{amsmath}
\usepackage[]{amssymb}
%amsthm, for theorem style modification
\usepackage{amsthm}
%column seperation 
\usepackage{multicol}
%blind text
\usepackage{blindtext}
%headings and footnote
\usepackage{fancyhdr}
\pagestyle{fancy}
\fancyhf{}
\rhead{Diary}
\lhead{My}
\rfoot{Page \thepage}
%color
\usepackage[]{color}
\usepackage[]{xcolor}



%%%%%%%%%%%%%%%%%%%%%%%%%%%%%%%%%%%%%%%%%%%%%
\title{\textbf{\Huge{随想贴}}}
\author{Yuchen Hui\thanks{yuchen22314@gmail.com} 
\and Yuchen Xi\thanks{Corresponding author}
\and 惠语辰 \thanks{???@????.???}}
\date{\today}

\begin{document}
    \maketitle
    \tableofcontents
    \listoftables
    \listoffigures

    \newpage

    \part{前言}
    今天是2022年3月2日。自从2021年,我已经很久没有写过日记一类的记录自己感想的
    文字了。但近日胸中颇有烦闷,不发不快,又想到\LaTeX 提供的字体格式均为我所爱,
    遂即兴创作了这个树洞式的\LaTeX “随想”文档以排解忧虑。
    
    2019年的时候我曾经在
    贴吧上发文叙述自己心中所想,终教人发现。念及贴吧旧情,我又将文档的标题改为“随想
    贴”。若以后偷得闲时,必将贴吧帖子悉数移至此文档!
    
    此文档不求日更,周更,甚至月更年更,其目的也并非供他人阅读。诚如贴吧所叙,只是
    为了纪念自己。考虑到身处异国他乡,纸稿易于遗失,我将这文档放在github上,借其力
    作一备份。如此一来,电脑中,硬盘里,网络上便是三重保障,文稿不至于刹那间尽失。我的
    github账号本无旧友知晓,新友亦不会有兴趣(时间)
    窥探我github账号的内容,而我又将文档匿于角落,故想来无人能发现。
    
    什么,你说如果看见了怎么办?那就看见了呗。我又不是要面子的正经人,正经人谁写日记啊。
    \begin{flushright}
        ——2022年3月2日,Montréal
    \end{flushright}
    \part{随笔}
    \section{2022年3月2日,雪 -18°C$\sim$~-6°C}。
    ε=(´ο`*)))唉。嘿嘿,我要睡了。tianyu同志还没回我。
    对于人际关系的处理,我一直
    诚惶诚恐。一方面,对于他,数学博士生,我确实很敬佩也很欣赏,
    而且看得出来我们的三观算是统一,经历也有许多可谈的,遇见他真是一个惊喜。
    我也早就把他看作我的好友。另一方面,升学的压力总是笼罩着我。虽然我现在
    成绩还行,申个本校那是百申百中,但总也想在本科期间有一段科研的经历。
    我现在要么去找stefan monnier,要么去找他,这是两个最简单也是最靠谱的法子,
    除了他俩我还真没认识的人。付强说过可以带我,不过那是一年前的事情了,
    现在也不好再去联系了。当然,这实在是怪我太懒惰,不愿打听哪里可以找科研做,
    以至于把希望寄托在这几个人身上。但更让我担心的是,友情因为这些事情而变味。
    好朋友之间的地位应该是平等的,因此我与他说话本应顺从天性而为,但由于害怕得罪了
    他失去了internship的机会,我发消息却总是小心翼翼。你看,因此今晚我也不便再发些“在吗”
    之类的话。每当我小心翼翼地考虑该怎么说话的时候,我都非常厌恶自己。我本不是这样
    唯利是图的人啊。我也真的是把他当做好朋友看待啊。要解决这一切的矛盾,最根本的
    当然还是从自身开始做起,积极向上一点不要再懒惰下去了。

    当然,他自己也说过“我的价值对你不就是……?”这种将人瞬间拉回物质世界的话,但我
    总觉着交友不能这样势利,最主要还是是依据对方的三观人品而定的。交真的朋友不是
    价值交换,而是一种相互的认可。诚然,价值交换能够巩固友情,但是我们还是要抓住主要矛盾。

    我发现自己总是以价值交换来思考问题,比如我和赖浩泽之间,不就是我提供c语言的资源,
    他带我打球么?这样看来,人便都是无情的,然而却忽略了我是看在他是嫡系学弟的关系上
    感情用事而主动帮助他,而并未为了让他带我打球而主导进行利益交换。(当然,他带我打球
    是否出于利益交换,或者出于感情,我不得而知。)张工请我吃饭,难道是为了我教他打球么?
    不会的。我帮助tianyu同志
    搞数据结构,一开始也许不愿意,但后来聊得投机,也就认真起来,也总不能说是为了利益交换
    而做。

    高逸然几年前说我没长大,我看也是。这些都是显然的道理,我却总是在纠结。须知社会
    本是你死我活的修罗场。不过有谁不想永远过着孩童般纯真快乐的生活呢?

    因此,看待问题的方法是这样:真的好朋友,那就主动帮助。对对方没感觉甚至厌恶,
    那还是利益交换吧。别再想其他的了。睡吧。没太多时间写其他的了。byebye咯。
    \section{2022年3月2日}
    正好在等电话,随便写点。今年魁北克的冬天真冷,连续一个月零下,最低气温更是
    跌倒过-30。要知道,19-20冬季最低才-18!而-20在今冬成为常态。

    暴雪也有好几场,虽然期间温度回到过零上,但总的来说能够保持两个月的积雪覆盖。

    海冰方面先扬后抑,年初白令海进展神速为十几年来最快,哈德孙湾冻结的较迟,却也
    在一个月的时间内快速完全封冻。可惜的是后冬鸡窝一直挤在加拿大和格陵兰岛这里,
    不但使得鄂霍茨克海冰龟速增长,更使中国发生暖冬,最终混了个不伦不类。
    看来得到了北美东北部的极寒,所付出的代价还是很大的。 后来我发
    现上个世纪有时白令海也能如此迅速地结冰,始知今冬海冰也没有过人之处。
    
    这是我第三年观察冬天的进展了。我已经差不多熟悉了各种天气套路,这项活动也该
    到此为止了。好好收心卷吧。

    \part{骂人}
\end{document}
